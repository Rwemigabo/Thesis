%*****************************************
\chapter{Background and Related work}\label{ch:methods}
%*****************************************
In this chapter, a review of the literature relevant to the development process of this application and other relevant terms to help with further understanding of this project are to be presented. Further more, we take a look at the MODA Clouds project, which is another could application automation system, which uses the MAPE strategy and a few other systems developed that implement and test some of the individual components developed in this system.

\section{Background }\label{sec:custom}
This section covers some background knowledge into the work done in the research areas related to this work, including terms constantly used throughout this report, which help to drive this project.

\subsection{ Topologies}
The concept an application topology, which based on the article The optimal distribution of applications in the cloud \cite{andrikopoulos2014optimal}  in summary is a labelled graph with a set of nodes, edges, labels and, source and target functions. It introduces and explains in full detail the concept an application's topology. A Topology can be referred to as a \textmu-Topology, which is split into  \textalpha-Topologies,  and \textgamma-Topologies, concepts which are also explained in the article mentioned above. The article also introduces the concept of viable topologies. The focus of this section will be on making clear what the \textalpha-Topologies and the Viable topologies are as these are the concepts most relevant to the system's development. Therefore, these are discussed further below including the relation to the project.

\subsubsection{\textalpha-Topologies}
As can be seen from an example in the article \cite{andrikopoulos2014optimal}, a type graph for a viable application topology can be referred to as a \textmu-Topology and therefore, the \textalpha-Topology is the application specific subgraph of the \textmu-Topology, which refers to the general application architectural setup.
The relation and importance of the above to the target functionality of the system being developed, which is to automatically switch between the available viable topologies provided by the user, is that it provided knowledge on what the system should be able to expect as input in terms of the topologies that will be used to determine the best option for the optimisation of the application the system is run on.
%% would include information like the number containers per service in the 

\subsubsection{Viable Topologies}
Knowing about the \textalpha-Topologies, it becomes clear what the term viable topologies refers to in the context of this project, are the different suitable topology options that will be available to the automation system being developed in order for it to be able to select the best topology option, which follows the set Service Level agreements and therefore is the best option for the particular application usage scenario. The concept of viable topologies is important for the development process of this automation project because it is a requirement for the system that the application it will be run on should have a number of viable topologies defined by the system architect, which will be the topologies the system switches between to optimise the resource usage of the application.

\subsection{Auto-scaling (Autonomous Computing)}
Different researchers have delved into a number of projects using different strategies to try and solve a variety of problems in the field of automation. These different projects range from: \textit{The monitoring of different metrics of the system}, For example whether to monitor the lower level metrics like the memory, cpu usage or even the network statistics or the higher level metrics like the rate of response to requests of the system being optimised. \textit{To the type of analysis strategy used to determine whether to scale up or down} for example the use of a predictive methods (see \cite{loff2014vadara}), reactive or rule based approaches (see \cite{amazon}) or hybrid methods using both reactive and predictive approaches. These different projects and automation strategies are discussed in a survey \cite{qu2018auto}, which helped provide an insight into the different options available to help with the completion of the different components of the system. The strategies used in the project will be looked into in a later chapter.

\subsection{Control Loops (MAPE-K)}
Throughout the various papers discussed in the survey \cite{qu2018auto} in the previous chapter, it is noticeable that the MAPE control loop strategy is a commonly used automation strategy, which involves the use four main procedures that make up this strategy, which are Monitor, Analyse, Plan and Execute. The MAPE automation strategy \cite{kephart2003vision} used for this project is complimented with a Knowledge base, which all the MAPE components interact with in order for the system to make an informed optimisation decision. These MAPE control loops including the knowledge base are the core driver of the system being developed as they are what makes up the complete functionality of the system, which are further looked at in a following chapter including the implementation process of the autonomous system.

\subsection{Microservices Architecture}
Microservices as discussed by Martin Fowler \cite{Microservices} describes an architecture style of building systems into a suite of smaller services each running on their own. These may be written in different programming languages and use different data storages. The microservice architecture is the architecture style focus for this project in terms of applications that the developed autonomous system will be able to perform its optimisation services on. The isolated services in this architecture style make it possible for the developed system to be able to individually run its loops on each service and therefore in the end perform the full assessment of the whole system, therefore performing its optimisation more efficiently and it's because of this characteristic that the Microservice style architecture was selected for this project. Additionally, the microservice style architecture is also well supported by most of the containerisation engines. This is an advantage in the case of this project since the popularity of containers among cloud developers recently has increased and therefore I was able to easily find a number of test applications that have the micro service architecture style and were deployed in containers especially the one selected for this project (Docker), which is introduced and shown later in a following chapter. 

\subsection{Fuzzy logic}
Fuzzy logic defn\cite{zadeh1996fuzzy}.......... Inspiration for the use of the method: find paper on survey and \cite{}. Connection / relevance to project and the outro

\section{Related work}
- cloud specific automation
\subsection{MODA Clouds}
Other work that has been done similar to this and a more specified look into the MODA Clouds project.
\subsection{Other work}
\cite{qu2018auto}







%*****************************************
%*****************************************
%*****************************************
%*****************************************
%*****************************************
