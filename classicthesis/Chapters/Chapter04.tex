%*****************************************
\chapter{Implementation}\label{ch:implementation}
In this chapter, the details of the implementation of the system are presented including a description of the individual  components of the system and how they interact, create and share data, some of the technologies that were helpful and necessary to the implementation of the system's components including their use to the system.

\section{Technologies and their utilisation in the system}
\subsection{Containerisation Technology (Docker)}
Containerisation\cite{pahl2017cloud} is a lightweight virtualisation technique, and virtualisation is necessary for the deployment of applications on the cloud.Therefore, when starting the development of this system, I had to select a containerisation engine where I could access the necessary metrics and upon which to deploy the test cloud applications given that the aim of the autonomous system would be to manage the resources of an application on the cloud so, I had to simulate this environment. There are a few containerisation technologies like rkt\cite{rkt} developed by CoreOS, Solaris Containers\cite{solaris} developed by Oracle and so on. However, given that I have some past experience with the Docker Engine, and additionally, the significant number of applications developed for and deployed with the docker engine, the choice was quite clear on what containerisation technology I should use for my project.
\subsubsection{Docker API}
While I had used the Docker engine before for the deployment of an application, I had not used it for the development of a system and therefore, I had to looked into its Application Programming Interface (API). From the page \cite{dockerAPI}, I found that it has a Software Development Kit(SDK) for Python and Go, and number of unofficial libraries for a number of programming languages, which opened up my options on the programming language I could use and enforced my confidence in the selection that had been made.
\subsubsection{Docker Compose}
One additional advantage of the docker engine is its easy compatibility with the Micro service architecture\cite{Microservices}. Docker Compose\cite{docker_compose}, a tool for running applications with multiple containers provides this feature when defining the different components in the docker-compose.yml file during the set up of the application. Given that, a number of micro service applications have been developed and deployed on docker, which provided me with a number of options both simple and complex for the testing phase of my system's components.
\subsection{Programming Language (Java)}
After deciding on the containerisation technology to use, it was time to move on to the programming language and my extensive knowledge in the Java as compared to other languages was a starting point. Next on the checklist was the compatibility with Docker and as seen from \cite{dockerAPI}, there is an official library for Java, which added to my confidence in the choice. Finally, the extensive number of options available to me in terms of what database to use also pushed my decision towards the java programming language. 
\subsection{Database (Relational)}
The option for a database was between either a Relational database or a NoSQL database and I was drawn towards the use of a relational database because of more previous experience with relational databases and they would properly accommodate the data that would be passed through the system.
\subsection{Fuzzy Logic (jFuzzyLogic)}
jFuzzyLogic is an opera source Java Library for Fuzzy Logic, which was introduced in \autoref{ch:background}. The Library was created to aid in the programming of Fuzzy Logic control systems using the standard Fuzzy Control language defined in the IEC 61131. Using the paper \cite{cingolani2013jfuzzylogic}, I was able to get some understanding of the library and some useful knowledge on how to use it in my project's development and specifically the analysis component.
\subsection{Locust IO}
\cite{locustIO}
\section{Application Components}
In this section, I discuss the individual system components of the system. Using \cite{iglesia2015mape} as a resource during the planning development phase of this system, I was able to implement the components even if some of the behaviour templates were not applicable with this system's domain.

\subsection{Sensor}
\begin{figure} [H]
   \centering 
   \includegraphics[scale=0.40]{gfx/Sensor_CD.png}
   \caption{ Sensor Component Class Diagram} 
   \label{Fig:2} 
\end{figure}

\subsection{Monitor}
\begin{figure} [H]
   \centering 
   \includegraphics[scale=0.40]{gfx/Monitor_CD.png}
   \caption{ Monitor Component Class Diagram} 
   \label{Fig:3} 
\end{figure}

\subsection{Analyse}
\begin{figure} [H]
   \centering 
   \includegraphics[scale=0.40]{gfx/Analyse_CD.png}
   \caption{ Analyse Component Class Diagram} 
   \label{Fig:4} 
\end{figure}

\subsection{Plan}
\begin{figure} [H]
   \centering 
   \includegraphics[scale=0.40]{gfx/Plan_CD.png}
   \caption{Plan Component Class Diagram} 
   \label{Fig:5} 
\end{figure}

\subsection{Execute}
\begin{figure}[H]
   \centering 
   \includegraphics[scale=0.50]{gfx/Execute_CD.png}
   \caption{Execute Component Class Diagram} 
   \label{Fig:6} 
\end{figure}




%*****************************************
%*****************************************
%*****************************************
%*****************************************
%*****************************************
