%*******************************************************
% Abstract
%*******************************************************
%\renewcommand{\abstractname}{Abstract}
\pdfbookmark[1]{Abstract}{Abstract}
\begingroup
\let\clearpage\relax
\let\cleardoublepage\relax
\let\cleardoublepage\relax

\chapter*{Abstract}
After years of advancement in Cloud Computing, including a significant increase in the number of Cloud service providers within a short period of time, application developers and enterprises have been left with a wide range of choice to fill their need for cloud services. Therefore, with this wide range of choices, one may find themselves making a choice of using a service that is cheaper on paper but ends up costing them more in the long run. 
\\In this project, a system is designed and implemented to cover one of the major concerns to application owners, which is the utilisation of the resources one is paying for. Resource utilisation can be one of the biggest costs to the owner of an application given that it can cost them one of two ways; by the loss of users if the application crashes due to lack of enough resources. And secondly, if the user has to over pay for resources that aren't being used by the application. The system developed uses the MAPE-K automation strategy proposed by IBM. It monitors the user's application and then analyses the application's usage statistics through the provisioning API and thereafter predicts what it's usage is going to be in the next time window. From that prediction, it make an adaptation plan if necessary by selecting a more suitable topology to handle the predicted load and finally redeploys the application with the more adequate Topology. The final system's functionalities are first tested on, a simple voting application and then evaluated using a larger web shop application. Both of these applications use the micro services architecture and the results of the testing and evaluation are to be presented.

\vfill

\endgroup

\vfill
